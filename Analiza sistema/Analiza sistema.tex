\documentclass{article}
\usepackage[utf8]{inputenc}
\usepackage{amsthm,amssymb,amsmath}
\usepackage{spverbatim}
\usepackage{graphicx}

\graphicspath{ {./images/} }
\newtheorem*{theorem}{Theorem}
\newcommand{\NN}{\mathbb{N}}
\newcommand{\ZZ}{\mathbb{Z}}
\newcommand{\RR}{\mathbb{R}}
\newcommand{\QQ}{\mathbb{Q}}
\newcommand{\CC}{\mathbb{C}}


\begin{document}
	\begin{titlepage}
		\newcommand{\HRule}{\rule{\linewidth}{1mm}} 
		\noindent
		{\large
		\begin{minipage}{0.2\textwidth}
				\begin{center} 
					\includegraphics[width=0.7\textwidth]{unsa.jpg}
				\end{center}
			\end{minipage}
			\begin{minipage}{0.58\textwidth}
				\begin{center} \large
					Univerzitet u Sarajevu\\
					Elektrotehnički fakultet u Sarajevu\\
					Odsjek za računarstvo i informatiku\\
				\end{center}
			\end{minipage}
			\begin{minipage}{0.2\textwidth}
				\begin{center} 
					\includegraphics[width=0.7\textwidth]{ETF_logo.png}
				\end{center}
			\end{minipage}
			\\[6 cm] 
			\begin{minipage}{0.2\textwidth}
				\begin{center} 
					
				\end{center}
			\end{minipage}
		
			
			


\begin{center}
	\vspace{0.1cm}
	\Large \textbf{Projektni zadatak}
	\vspace{0.1cm}
\end{center}

\rule{\textwidth}{0.1mm}

\begin{center}
	\vspace{0.1cm}
	\LARGE \textbf{Studentski dom}
	\vspace{0.3cm}
\end{center}

\rule{\textwidth}{0.1mm}

\begin{center}
    \textbf{Dženana Huseinspahić, Esma Karahodža, \\Dženeta Kudumović}\\
   \textit{Objektno orjentisana analiza i dizajn}\\
    Akademska godina: 2019/2020\\
    \vspace{0.5cm}
\end{center}

%Sažetak


\selectlanguage{croatian}

\vfill
\begin{center}
	Sarajevo, maj 2020.
\end{center}	 		
		}
	\end{titlepage}


%%%%%%%%%%%%%%%%%%%%%%%%%%%%%%%%%%%%%%%%%%%%%%%%%%%%%%%%%%%%%%%%%%%%%%%%%%%%%%
\renewcommand{\contentsname}{Sadržaj}
\tableofcontents

\newpage

\section{Opis projekta}
\hspace{\indent}Na svim većim univerzitetima širom svijeta veliki broj studenata živi u studentskim domovima. Namjena ovog softvera je da studentima olakša izvršavanje svakodnevnih aktivnosti u domu i da uposlenicima pomogne da što efikasnije izvršavaju neke od resursno zahtjevnih poslova i studentskih servisa. Na taj način se automatiziraju određeni procesi koji oduzimaju dosta vremena studentima i osoblju koje ih izvršava.

\subsection*{Funkcionalnosti}
\begin{itemize}
  \item Prijem studenata u studentski dom – popunjavanje aplikacije, kontrola dokumentacije, odobravanje zahtjeva
  \item Održavanje informacija o studenatima - lični podaci i osnovne informacije vezane za servise u studentskom domu
  \item Dnevna rezervacija obroka u kuhinji od strane studenata
  \item Prijava zahtjeva za tekućim održavanjem studentskih apartmana i zajedničkih prostorija
  \item Upravljanje zahtjevima za održavanje – dodjela zadataka tehničkom osoblju, evidentiranje završetka poslova i izvještavanje
  
\end{itemize}

\subsection*{Akteri}
\begin{itemize}
    \item Student
    \item Uposlenik uprave
    \item Šef kuhinje
    \item Šef odjela tehničkog održavanja
\end{itemize}


\medskip 

\newpage

%%%%%%%%%%%%%%%%%%%%%%%%%%%%%%%%%%%%%%%%%%%%%%%%%%%%%%%%%%%%%%%%%%%%%%%%%%%%%%
\section{Klase}

\subsection*{Student}
\begin{itemize}
  \item atributi 
     \begin{itemize}
        \item imeIPrezime: String
        \item JMBG: String
        \item adresaStanovanja: String
        \item fakultet: String
        \item godinaStudiranja: int
        \item brojTelefona: int
        \item brojSobe: int
        \item email: String
        \item brojIndeksa: int
        \item brojBonova: int
     \end{itemize}
  \item metode
  \begin{itemize}
        \item \(<<\)create\(>>\)Student(in s:String, in jmbg:String, in a:String, in f:String, in g:int, in br:int, in brS:int, in e:String, in i:int, in b:int)
     \end{itemize}
\end{itemize}

\subsection*{StudentskiDom}
\begin{itemize}
  \item atributi 
     \begin{itemize}
        \item studenti: List<Student>
     \end{itemize}
  \item metode
  \begin{itemize}
        \item \(<<\)create\(>>\)studentskiDom()
        \item prijava(in student:Student): void
        \item odjava(in student:Student): void
     \end{itemize}
\end{itemize}

\subsection*{PrijavaObroka}
\begin{itemize}
  \item atributi 
     \begin{itemize}
        \item student: Student
        \item rucak: Boolean
        \item vecera: Boolean
        \item zaPonijet: Boolean
     \end{itemize}
  \item metode
  \begin{itemize}
        \item \(<<\)create\(>>\)prijavaObroka(in s:Student)
        \item smanjiBonove(): void
        \item dodajPrijavu(in s:Student): PrijavaObroka
     \end{itemize}
\end{itemize}

\subsection*{PrijavaUDom}
\begin{itemize}
  \item atributi 
     \begin{itemize}
        \item student: Student
        \item vrijemePrijave: Date

     \end{itemize}
  \item metode
  \begin{itemize}
        \item \(<<\)create\(>>\)prijavaUDom(in s:Student, in d:Date)
     \end{itemize}
\end{itemize}

\subsection*{PrijavaKvara}
\begin{itemize}
  \item atributi 
     \begin{itemize}
        \item student: Student
        \item tipKvara: enum
        \item opisKvara: String
        \item vrijemeKvara: Date

     \end{itemize}
  \item metode
  \begin{itemize}
        \item \(<<\)create\(>>\)prijavaKvara(in s:Student, in t:enum, in o:String, in v:Date)
        \item dodajPrijavu(in s:Student): PrijavaKvara
     \end{itemize}
\end{itemize}

\subsection*{UposlenikDoma}
 
\begin{itemize}
  \item interface
  \item atributi
     \begin{itemize}
        \item imeIPrezime: String
     \end{itemize}
\end{itemize}
 
\subsection*{SefKuhinje}
 
\begin{itemize}
  \item izvedena klasa iz klase UposlenikDoma
  \item atributi
     \begin{itemize}
        \item obroci: List\(<\)PrijavaObroka\(>\)
     \end{itemize}
  \item metode
  \begin{itemize}
        \item \(<<\)create\(>>\)sefKuhinje()
        \item odobriPrijave(): void
       
     \end{itemize}
 
\end{itemize}
 
\subsection*{UposlenikUprave}
 
\begin{itemize}
  \item izvedena klasa iz klase UposlenikDoma
  \item atributi
     \begin{itemize}
        \item prijave: List\(<\)PrijavaUDom\(>\)
     \end{itemize}
  \item metode
  \begin{itemize}
        \item \(<<\)create\(>>\)uposlenikUprave(s: String)
        \item odobriPrijavuUDom(prijava: PrijavaUDom): Boolean
        \item promijeniPodatkeOStudentu(student: Student): void
        \item unesiStudenta(student: Student): void
       
     \end{itemize}
 
\end{itemize}
 
\subsection*{SefTehnickogOdrzavanja}
 
\begin{itemize}
  \item izvedena klasa iz klase UposlenikDoma
  \item atributi
     \begin{itemize}
        \item radnici: List\(<\)Radnik\(>\)
        \item kvarovi: List\(<\)PrijavaKvara\(>\)
     \end{itemize}
  \item metode
  \begin{itemize}
        \item \(<<\)create\(>>\)sefTehnickogOdrzavanja()
        \item pronadjiRadnika(prijava: PrijavaKvara): PrijavaKvara
       
     \end{itemize}
 
\end{itemize}
 
 
 
\subsection*{Radnik}
 
\begin{itemize}
  \item izvedena klasa iz klase SefTehnickogOdrzavanja
  \item atributi
     \begin{itemize}
        \item Usluga: enum
        \item brojTelefona: int
     \end{itemize}
  \item metode
  \begin{itemize}
        \item \(<<\)create\(>>\)radnik(usluga: enum, brojTelefona: int)
       
     \end{itemize}
 
\end{itemize}



\newpage

%%%%%%%%%%%%%%%%%%%%%%%%%%%%%%%%%%%%%%%%%%%%%%%%%%%%%%%%%%%%%%%%%%%%%%%%%%%%%%
\section{Solid principi}

\subsection{Single Responsibility Principe} 

\hspace{\indent}Princip S zahtijeva da svaka klasa ima samo jednu odgovornost, odnosno da klasa vrši samo jedan tip akcija kako ne bi ovisila o prevelikom broju konkretnih implementacija. 

U ovom projektu princip S je ispoštovan jer svaka klasa vrši samo jednu vrstu akcija, npr. postoje tri različite klase za različite prijave umjesto jedne koja bi slala podatke svim klasama. 

\subsection{Open/Closed Principle}

\hspace{\indent}Princip O zahtijeva da klasa koja koristi neku drugu klasu ne treba biti modificirana pri uvođenju novih funkcionalnosti, ili pri potrebi za mijenjanjem druge klase.

Vidimo da za sve klase na klasnom dijagramu vrijedi da možemo mijenjati okruženje oko klase bez promjene same klase. Npr. na našem klasnom dijagramu nijedna klasa me zavisi od neke druge klase na način da bi promjenom te druge klase morali modificirati i prvu.

\subsection{Liskov Substitution Principle}

\hspace{\indent}Princip L zahtijeva da nasljeđivanje bude ispravno implementirano, odnosno da je na svim mjestima na kojima se koristi osnovni objekat moguće iskoristiti i izvedeni objekat a da takvo nešto ima smisla.

Liskov princip je ovdje zadovoljen jer se sve naslijeđene klase mogu zamijeniti svojim osnovnim tipom, npr. klasu šefKuhinje možemo zamijeniti klasom Uposlenik Doma bez da stvorimo konflikt. 

\subsection{Interface Segregation Principle} 

\hspace{\indent}Princip I zahtijeva da i svi interfejsi zadovoljavaju princip S, odnosno da svaki interfejs obavlja samo jednu vrstu akcija.

Klasni dijagram sadrži samo jedan interfejs, UposlenikDoma. On obavlja samo jednu akciju a to je da povezuje tri vrste uposlenika doma. Zbog toga je i ovaj princip zadovoljen.



\subsection{Dependency Inversion Principle}

Princip D zahtijeva da pri nasljeđivanju od strane više klasa bazna klasa uvijek bude apstraktna. Razlog za ovo je što je teško koordinisati veliki broj naslijeđenih klasa i konkretnu baznu klasu ukoliko ista nije apstraktna, a da pritom kod bude čitak i jednostavan za razumijevanje.

Ovaj projekat ne sadrži nijednu baznu klasu koja nije apstraktna i samim time je ovaj princim automatski ispunjen.

\newpage

\section{Paterni}

\subsection{Strukturalni paterni}

\subsubsection*{Proxy}
\hspace{\indent}Proxy patern služi za dodatno osiguravanje objekata od pogrešne ili zlonamjerne upotrebe. Primjenom ovog paterna omogućava se kontrola pristupa objektima, te se onemogućava manipulacija objektima ukoliko neki uslov nije ispunjen, odnosno ukoliko korisnik nema prava pristupa traženom objektu.

Proxy patern je korišten kako bi samo dređeni stanovnici mogli pristupati određenim dijelovima sistema, np. ne bi trebalo da šef kuhinje odobri prijave u dom ili da korisnik doma pristupi informacijama o drugom korisniku doma. U klasi Proxy svakom korisniku dodjeljujemo odgovarajući nivo pristupa, a zatim korisnik može pristupati odgovarajućim metodama iz klase IStudentskiDom.


\subsubsection*{Composite}
\hspace{\indent}Composite patern služi za kreiranje hijerarhije objekata. Koristi se kada svi objekti imaju različite implementacije nekih metoda, no potrebno im je svima pristupati na isti način, te se na taj način pojednostavljuje njihova implementacija.

U našem projektu je postojala apstraktna klasa Prijava koja je zamijenjena interfejsom IPrijava koji su sve 3 vrste prijava naslijedile i tako je hijerarhija pojednostavljena.

\subsection{Kreacijski paterni}

\subsubsection*{Singleton}

\hspace{\indent}Singleton patern služi kako bi se neka klasa mogla instancirati samo jednom. Na ovaj način može se omogućiti i tzv. lazy initialization, odnosno instantacija klase tek onda kada se to prvi
put traži. Osim toga, osigurava se i globalni pristup jedinstvenoj instanci - svaki put kada joj
se pokuša pristupiti, dobiti će se ista instanca klase. Ovo olakšava i kontrolu pristupa u slučaju
kada je neophodno da postoji samo jedan objekat određenog tipa.

U našem projektu singleton patern je iskorišten u klasi koja simulira bazu podataka i koja se kreira samo jednom.


\subsubsection*{Builder}

\hspace{\indent}Builder patern služi za apstrakciju procesa konstrukcije objekta, kako bi se kao rezultat
mogle dobiti različite specifikacije objekta koristeći isti proces konstrukcije. Ovaj patern koristi
se kako bi se izbjeglo kreiranje kompleksne hijerarhije klasa te kako bi se izbjegao kompleksni
programski kod konstruktora jedne klase koja može imati različite konfiguracije atributa.
Različiti dijelovi konstrukcije objekta izdvajaju se u posebne metode koje se zatim pozivaju
različitim redoslijedom ili se poziv nekih dijelova izostavlja, kako bi se dobili željeni različiti
podtipovi objekta bez potrebe za kreiranjem velikog broja podklasa.

Builder paternt je ovdje iskorišten tako da se pri kreiranju studenta razlikuje prijava u kojoj student ima prijavu za cimeraj od studenta kome će se nasumično dodijeliti soba.


\newpage

%%%%%%%%%%%%%%%%%%%%%%%%%%%%%%%%%%%%%%%%%%%%%%%%%%%%%%%%%%%%%%%%%%%%%%%%%%%%%%
\section{MVC}




\end{document}
